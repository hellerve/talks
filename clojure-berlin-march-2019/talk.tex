\documentclass[aspectratio=169]{beamer}
\usepackage{minted}
\usepackage{listings}
\usetheme{metropolis}
\title{Leveraging macros to reimplement language features}
\date{\today}
\author{Veit Heller}
\institute{Clojure Meetup Berlin}
\begin{document}
  \maketitle
  \begin{frame}{Agenda}
    \begin{itemize}
      \item Comments
      \item Conditionals
      \item Lists
      \item Local bindings
    \end{itemize}
  \end{frame}
  \begin{frame}{Agenda}
    Shameless plug: agenda of my series of blog posts on Scheme macros!
    \begin{itemize}
      \item Modules
      \item Generic functions
      \item Classes/Inheritance
      \item Keyword Arguments in functions
      \item Design by Contract
      \item Green Threads
    \end{itemize}
  \end{frame}
  \section{Comments}
  \begin{frame}[fragile]
    \frametitle{Comments}
    \begin{listing}[H]
      \caption{Implementing comments}
      \begin{minted}{clojure}
(defmacro mcomment [& args] nil)
      \end{minted}
    \end{listing}
  \end{frame}
  \begin{frame}[fragile]
    \frametitle{Comments}
    \begin{listing}[H]
      \caption{Using comments}
      \begin{minted}{clojure}
(mcomment
  (+ 40 3)
  (undefined-function)
  (println "nope"))
      \end{minted}
    \end{listing}
  \end{frame}
  \section{Conditionals}
  \begin{frame}[fragile]
    \frametitle{if}
    \begin{listing}[H]
      \caption{Booleans as functions}
      \begin{minted}{clojure}
(defn mtrue [x y] (x))
(defn mfalse [x y] (y))
      \end{minted}
    \end{listing}
  \end{frame}
  \begin{frame}[fragile]
    \frametitle{if}
    \begin{listing}[H]
      \caption{\texttt{if} as a macro}
      \begin{minted}{clojure}
(defn mtrue [x y] (x))
(defn mfalse [x y] (y))

(defmacro mif [cnd then else] `(~cnd #(do ~then) #(do ~else)))
      \end{minted}
    \end{listing}
  \end{frame}
  \begin{frame}[fragile]
    \frametitle{if}
    \begin{listing}[H]
      \caption{Using \texttt{if}, the macro}
      \begin{minted}{clojure}
(mif mfalse
  (println "nope")
  (println "yup"))
      \end{minted}
    \end{listing}
  \end{frame}
  \begin{frame}[fragile]
    \frametitle{cond}
    \begin{listing}[H]
      \caption{\texttt{cond} as a macro}
      \begin{minted}{clojure}
(defmacro mcond [& args]
  (when args
    `(if ~(first args)
       ~(second args)
       ~(cons 'mcond (rest (rest args))))))
      \end{minted}
    \end{listing}
  \end{frame}
  \begin{frame}[fragile]
    \frametitle{cond}
    \begin{listing}[H]
      \caption{Using \texttt{cond}, the macro}
      \begin{minted}{clojure}
(mcond
  (= 1 2) (println "no")
  (= 2 3) (println "still no")
  true (println "yup"))
      \end{minted}
    \end{listing}
  \end{frame}
  \section{Lists}
  \begin{frame}[fragile]
    \frametitle{cons, car, cdr}
    \begin{listing}[H]
      \caption{\texttt{cons}, \texttt{car}, and \texttt{cdr} as functions}
      \begin{minted}{clojure}
(defn mcons [h t] #(if % h t))
(defn mcar [l] (l true))
(defn mcdr [l] (l false))
      \end{minted}
    \end{listing}
  \end{frame}
  \begin{frame}[fragile]
    \frametitle{cons, car, cdr}
    \begin{listing}[H]
      \caption{Using \texttt{cons}, \texttt{car}, and \texttt{cdr}, the macros}
      \begin{minted}{clojure}
(println (mcar (mcdr (mcons 1 (mcons 2 3)))))
      \end{minted}
    \end{listing}
  \end{frame}
  \begin{frame}[fragile]
    \frametitle{list}
    \begin{listing}[H]
      \caption{Reimplementing \texttt{list}}
      \begin{minted}{clojure}
(defn mlist [& args]
  (loop [res nil, args args]
    (if (empty? args)
      res
      (recur (mcons (last args) res) (butlast args)))))
      \end{minted}
    \end{listing}
  \end{frame}
  \begin{frame}[fragile]
    \frametitle{cons, car, cdr}
    \begin{listing}[H]
      \caption{\texttt{mquoted}}
      \begin{minted}{clojure}
(defmacro mquoted [& args] `(apply mlist '~args))
      \end{minted}
    \end{listing}
  \end{frame}
  \section{Local bindings}
  \begin{frame}[fragile]
    \frametitle{let}
    \begin{listing}[H]
      \caption{Reimplementing \texttt{let}}
      \begin{minted}{clojure}
(defmacro mlet [args body]
  (if (= (count args) 2)
    `((fn [~(first args)] ~body) ~(second args))
    `((fn [~(first args)]
        (mlet ~(rest (rest args)) ~body))
      ~(second args))))
      \end{minted}
    \end{listing}
  \end{frame}
  \begin{frame}[fragile]
    \frametitle{let}
    \begin{listing}[H]
      \caption{Using \texttt{let}}
      \begin{minted}{clojure}
(println (mlet [x 1 y 2] (+ x y)))
      \end{minted}
    \end{listing}
  \end{frame}
  \begin{frame}{References}
    \begin{itemize}
      \item “Reversing the technical interview” uses the lists for great effect: \texttt{https://aphyr.com/posts/340-reversing-the-technical-interview}
      \item These slides: \texttt{https://github.com/hellerve/talks}
      \item A series of blog posts on Scheme macros: \texttt{https://blog.veitheller.de/scheme-macros}
    \end{itemize}
  \end{frame}
  \begin{frame}{The End}
    \Huge Thank you!
    \linebreak
    \linebreak
    \linebreak
    \small Questions?
    \linebreak
    \linebreak
    \tiny Slides at \texttt{https://github.com/hellerve/talks}
  \end{frame}
\end{document}
