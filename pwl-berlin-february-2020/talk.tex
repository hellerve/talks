\documentclass[aspectratio=169]{beamer}
\usepackage{minted}
\usepackage{listings}
\usetheme{veit}
\title{SCHEME: An Interpreter for Extended Lambda Calculus}
\subtitle{Gerald J. Sussman and Guy L. Steele Jr.}
\date{\today}
\author{Veit Heller}
\institute{Papers We Love Berlin}
\begin{document}
  \maketitle
  \begin{frame}{Agenda}
    \begin{itemize}
      \item Introduction and historical context
      \item Scheme primer
      \item The good stuff
      \item Let’s see some code!
      \item Implementation notes
    \end{itemize}
  \end{frame}
  \section{Introduction}
  \begin{frame}{The Paper}
    In 1975, a 21-year-old grad student named Guy Steele and his thesis
    advisor Gerald Sussman had something to show to the world: a little
    programming language called Scheme.
  \end{frame}
  \begin{frame}{The Paper}
    \begin{figure}
      \includegraphics[width=\linewidth]{header.png}
      \caption{A wild paper appears.}
    \end{figure}
  \end{frame}
  \begin{frame}{The Paper}
    The paper has all the goods a hacker could wish for: a reference, cool code
    examples, and an implementation of Lisp in Lisp.
  \end{frame}
  \begin{frame}{The Name}
    The language was originally intended to be called SCHEMER, in reference
    to its ancestors PLANNER and CONNIVER.
  \end{frame}
  \section{Scheme: A primer}
  \begin{frame}[fragile]
    \frametitle{Scheme: A primer}
    In Scheme, we define functions using \texttt{define}—you might know it as
    \texttt{defn} or \texttt{defun} in other Lisps:

    \begin{listing}[H]
      \caption{Defining addition}
      \begin{minted}{scheme}
(define add
  (lambda (x y)
    (+ x y)))
      \end{minted}
    \end{listing}

    NB: I eschewed the all-caps notation, and I hope your eyes will thank me
    for it.
  \end{frame}
  \begin{frame}[fragile]
    \frametitle{Scheme: A primer}
    We can \texttt{quote} things using either the function or the abbreviation
    \texttt{'<thing>}.

    \begin{listing}[H]
      \caption{Using symbols as values}
      \begin{minted}{scheme}
; this will always return the symbol x
(define gimme-x (lambda () 'x))
      \end{minted}
    \end{listing}
  \end{frame}
  \begin{frame}[fragile]
    \frametitle{Scheme: A primer}
    There is also the somewhat idiosyncratic \texttt{labels}, which allows you
    to define local functions that can be called inside a context, and can call
    themselves and other local functions in that context. You might know it as
    \texttt{letrec*} from later Schemes, and as simply \texttt{let} in Common
    Lisp.
  \end{frame}
  \begin{frame}[fragile]
    \frametitle{Putting it all together}
    \begin{listing}[H]
      \caption{Let’s define something!}
      \begin{minted}{scheme}
; lets define cells!
(define cons-cell (lambda (contents)
    (labels ((the-cell
                (lambda (msg)
                  (if (eq msg 'contents) contents
                    (if (eq msg 'cell?) 'yes
                      (if (eq (car msg) '<-)
                        (block (aset 'contents (cadr msg))
                               the-cell)
                        (error '|Unrecognized Message - Cell|
                               msg
                               'wrng-type-arg)))))))
      the-cell)))
      \end{minted}
    \end{listing}
  \end{frame}
  \begin{frame}{And now?}
    There is more, though!
  \end{frame}
  \section{The good stuff}
  \begin{frame}[fragile]
    \frametitle{Continuations!}
    \begin{listing}[H]
      \caption{Jump around aka. “Sussman’s favorite style/Steele’s least favorite”}
      \begin{minted}{scheme}
(define sqrt (lambda (x epsilon)
  ((lambda (ans looptag)
    (catch returntag ; setup return label
      (progn
        (aset 'looptag (catch m m)) ; setup loop label
        (if (< (abs (- (* ans ans) x)) epsilon)
          (returntag ans) ; goto return label
          nil) ; not done yet
        (aset 'ans (/ (+ (/ x ans) ans) 2.0)) ; calculate step
        (looptag looptag)))) ; goto loop label
    1.0
    nil)))
      \end{minted}
    \end{listing}
  \end{frame}
  \begin{frame}{Wait, what?}
    Continuations effectively allow us to pause and resume computations, to
    pretend to call a function but instead moving between different interpreter
    states.
  \end{frame}
  \begin{frame}{Wait, what?}
    It’s pretty mind-bending at first and I understand if it’s a little much.

    \bigskip

    \small The paper is a bit harsh here: “Anyone who doesn’t understand how
    this manages to work probably should not attempt to use CATCH.”
  \end{frame}
  \begin{frame}{Multiprocessing}
    As if that wasn’t enough, we also have a multiprocessing story. We can
    create new processes using \texttt{create!process}, start them using
    \texttt{start!process}, stop them using \texttt{stop!process}, and
    synchronize using \texttt{evaluate!uninterruptibly}.
  \end{frame}
  \begin{frame}{Oof.}
    “This concludes the SCHEME Reference Manual.”
  \end{frame}
  \section{Code Examples}
  \begin{frame}[fragile]
    \frametitle{Continuations, yet again!}
    \begin{listing}[H]
      \caption{Factorial, but the computation happens in continuations.}
      \begin{minted}{scheme}
(define fact (lambda (number continuation)
  (if (= number 0)
    (continuation 1)
    (fact (- number 1)
          (lambda (a) (continuation (* number a)))))))

; simple computation
(fact 5 (lambda (x) x))

; computation, but we log each step
(fact 5 (lambda (x) (block (print x) x)))
      \end{minted}
    \end{listing}
  \end{frame}
  \begin{frame}{Multiprocessing}
    Consider you have two functions, and you want to run them in parallel,
    stopping whenever the first one terminates, and returning its result.

    \bigskip

    The authors call this “A Useless Multiprocessing Example”.
  \end{frame}
  \begin{frame}[fragile]
    \frametitle{Multiprocessing}
    \begin{listing}[H]
      \caption{Dont!Shout!At!Me}
      \begin{minted}{scheme}
(define try!two!things!in!parallel (lambda (f1 f2)
  (catch c
    ((lambda (p1 p2)
      ((lambda (f1 f2)
        ; ensures both fs get started atomically
        (evaluate!uninterruptibly
          (block (aset 'p1 (create!process '(f1)))
                 (aset 'p2 (create!process '(f2)))
                 (start!process p1)
                 (start!process p2)
                 (stop!process **process**)))) ; stop yourself
          ; what are f1 and f2?
          ))
      nil nil))))
      \end{minted}
    \end{listing}
  \end{frame}
  \begin{frame}[fragile]
    \frametitle{Multiprocessing}
    \begin{listing}[H]
      \caption{The magic bits}
      \begin{minted}{scheme}
; f1 = 
(lambda ()
  ; stop the other process and return
  ((lambda (value)
    (evaluate!uninterruptibly
      (block (stop!process p2) (c value))))
   (f1))) ; do our thing
; f2 =
(lambda ()
  ((lambda (value)
    (evaluate!uninterruptibly
      (block (stop!process p1) (c value))))
   (f2)))
      \end{minted}
    \end{listing}
  \end{frame}
  \begin{frame}[fragile]
    \frametitle{Pattern matching!}
    Let’s consider a simple pattern matching function.
    \begin{listing}[H]
      \caption{A simple pattern matcher.}
      \begin{minted}{scheme}
; ! = zero or more things (.* in regex)
; ? = any single thing (. in regex)
; anything else = itself

(match '(A !B ?C ?C !B !E)
       '(A X Y Q Q X Y Z Z X Y Q Q X Y R))
      \end{minted}
    \end{listing}
  \end{frame}
  \begin{frame}{Pattern matching!}
    Instead of just returning the match groups, however, we return the match
    groups and a continuation that gives us backtracking and will return
    alternative matches, Prolog-style.

    \bigskip

    How would we implement this?
  \end{frame}
  \begin{frame}{Pattern matching!}
    \begin{figure}
      \includegraphics[height=7cm]{match.png}
      \caption{A simple solution to a simple problem.}
    \end{figure}
  \end{frame}
  \begin{frame}{Examples I wish I would have had time for}
    If you have the time to study the code examples, keep your eyes peeled
    for the definition of the \texttt{do} macro and the \texttt{samefringe}
    problem.
  \end{frame}
  \section{Implementation Notes}
  \begin{frame}
    TODO: Talk about the implementation.
  \end{frame}
\end{document}
