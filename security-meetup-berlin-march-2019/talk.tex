\documentclass[aspectratio=169]{beamer}
\usetheme{metropolis}
\title{Programming Languages and Trust}
\date{\today}
\author{Veit Heller}
\institute{Datengarten | CCCB}
\begin{document}
  \maketitle
  \begin{frame}{\texttt{whoami}}
    \begin{itemize}
      \item I work at a consultancy.
      \item I hack on languages in my free time.
      \item zepto, Carp, cspfuck\ldots
      \item I’m secretly a turtle.
    \end{itemize}
  \end{frame}
  \section{Trust}
  \begin{frame}{Trusting Trust}
    \begin{itemize}
      \item In 1984, Ken Thompson was rightfully awarded the Turing Award.
      \item He wrote a three-page paper on a scary idea: malicious compilers.
    \end{itemize}
  \end{frame}
  \begin{frame}{The idea}
    \begin{itemize}
      \item Have you heard about Quines? They are self-replicating programs.
      \item Have you heard about bootstrapping compilers? They are compilers
            that can compile themselves.
      \item What if we compile a “buggy” version of our compiler and ship it?
    \end{itemize}
  \end{frame}
  \begin{frame}{The idea}
    “The actual bug I planted in the compiler would match code in the UNIX
     "login" command.  The replacement code would miscompile the login
     command so that it would accept either the intended encrypted password
     or a particular known password. Thus if this code were installed in
     binary and the binary were used to compile the login command, I could
     log into that system as any user.”

     \indent — Ken Thompson, Reflections on Trusting Trust, page 3
  \end{frame}
  \begin{frame}{The idea}
    “Such blatant code would not go undetected for long. Even the most casual
     perusal of the source of the C compiler would raise suspicions.” 

     \indent — Ken Thompson, Reflections on Trusting Trust, page 3
  \end{frame}
  \begin{frame}{The idea}
    “This [second approach] simply adds a second Trojan horse to the one that
     already exists. The second pattern is aimed at the C compiler. [\ldots] First
     we compile the modified source with the normal C compiler to produce a
     bugged binary. We install this binary as the official C. We can now remove
     the bugs from the source of the compiler and the new binary will reinsert
     the bugs whenever it is compiled. Of course, the login command will remain
     bugged with no trace in source anywhere.”

     \indent — Ken Thompson, Reflections on Trusting Trust, page 3
  \end{frame}
  \begin{frame}{Takeaways}
    \begin{itemize}
      \item It has historically mostly been regarded as a scary practical joke
            by compiler engineers.
      \item As compilers get more complex (and more modular), a malicious
            “optimization” pass can ever more easily be inserted into the
            compiler.
      \item If someone pointed at an obscure piece of assembly and told you
            that “this simple optimization buys us a 10\% speed gain on
            auto-vectorized code in ARM”, would you just believe them?
    \end{itemize}
  \end{frame}
  \begin{frame}{Caveat}
    Nota Bene: There is a workaround described by David A. Wheeler, but it
    assumes that there is no non-determinism in a compiler, but there often is
    (see, for instance, GCC’s \texttt{-fguess-branch-probability}).

    Your best bet is probably reproducible builds.
  \end{frame}
  \section{Demo—enter Michael Arntzenius}
  \begin{frame}{References I}
    \begin{itemize}
      \item These slides: \texttt{https://github.com/hellerve/talks}
      \item Ken Thompson, Reflections on Trusting Trust: \texttt{https://www.win.tue.nl/~aeb/linux/hh/thompson/trust.html}
      \item Michael Arntzenius’ reflections on Trusting Trust: \texttt{https://github.com/rntz/rotten}
      \item David A. Wheeler, Fully Countering Trusting Trust through Diverse Double-Compiling (DDC) - Countering Trojan Horse attacks on Compilers: \texttt{https://dwheeler.com/trusting-trust/}
    \end{itemize}
  \end{frame}
  \begin{frame}{\texttt{exit}}
    \Huge Thank you!
    \linebreak
    \linebreak
    \linebreak
    \small Questions?
    \linebreak
    \linebreak
    \tiny Slides at \texttt{https://github.com/hellerve/talks}
  \end{frame}
\end{document}
