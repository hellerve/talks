\documentclass[aspectratio=169]{beamer}
\usetheme{metropolis}
\title{Abstractions! How do I even?}
\date{\today}
\author{Veit Heller}
\institute{EnthusiastiCon 2019}
\begin{document}
  \maketitle
  \begin{frame}{Agenda}
    \begin{itemize}
      \item Motivation!
      \item Michel Foucault—The Order of Things
      \item Robert M. Pirsig—Zen and the Art of Motorcycle Maintenance
      \item Summary!
    \end{itemize}
  \end{frame}
  \section{Motivation!}
  \begin{frame}{Technophilosophy?}
    We’re going to talk a lot about philosophy. None of it will be about the
    branches of philosophy that seem most notably intertwined with our
    profession: technophilosophy.
  \end{frame}
  \begin{frame}{Abstractions}
    We build abstractions, always.
  \end{frame}
  \begin{frame}{Abstractions}
    So do philosophers; they have a headstart of just a few millenia.
  \end{frame}
  \begin{frame}{Technophilosophy!}
    In a sense, all philosophy is philosophy about technology, if the goal of
    technology is abstracting things.
  \end{frame}
  \begin{frame}{But I don’t have a PhD!}
    Me neither, and English isn’t my first language.

    The beauty about a lot of philosophy is that it is very simple, because it
    deals with very simple things—but not easy things!
  \end{frame}
  \begin{frame}{Why do I have to think about these things?}
    Because you work on them all day! And because it’s fun!
  \end{frame}
  \section{Michel Foucault—The Order of Things}
  \begin{frame}{Disclaimer}
    This book is not about abstractions, but about “episteme”, which are roughly
    the fundamental ideological axioms of all of knowledge—think paradigms, but
    without knowing that you adhere to them. This doesn’t make any sense
    without context, and the talk isn’t really about it anyway.

    What it is about is how to impose order through abstraction.
  \end{frame}
  \begin{frame}{Jorge Luis Borges}
    “[\ldots] animals are divided into: (a) belonging to the Emperor, (b)
    embalmed, (c) tame, (d) sucking pigs, (e) sirens, (f) fabulous, (g) stray
    dogs, (h) included in the present classification, (i) frenzied, (j)
    innumerable, (k) drawn with a very fine camelhair brush, (l) \textit{et
    cetera}, (m) having just broken the water pitcher, (n) that from a long
    way off look like flies [\ldots]”

    — Michel Foucault, citing Jorge Luis Borges, in “The Order of Things”
  \end{frame}
  \begin{frame}{Order}
    It is often better to have no system than to have the wrong system.
  \end{frame}
  \begin{frame}{Order}
    Classifications are defined by the space between them, not by their label.
  \end{frame}
  \begin{frame}{Order}
    Order is statist: you have to impose rules, and all of them are arbitrary.
  \end{frame}
  \begin{frame}{What makes abstractions good? What makes them bad?}
    Good abstractions impose a sensible order. The space between concepts is
    well defined.
  \end{frame}
  \begin{frame}{Why order?}
    If the order is intuitive, the abstraction will be intuitive as well.
  \end{frame}
  \begin{frame}{Abstractions and Order}
    A good abstraction is good because it imposes order. Good order is good
    because it provides abstraction.

    This is a tautology, but it makes sense because this is how the mind
    operates: always classifying, putting in boxes, hiding details.
  \end{frame}
  \section{Robert M. Pirsig—Zen and the Art of Motorcycle Maintenance}
  \begin{frame}{Disclaimer}
    This book is good, and reading it is very addicting!

    It’s about a father-son motorcycle trip, a life changed by mental illness,
    and the nature of “Quality”.
  \end{frame}
  \begin{frame}{The Three Things}
    There are three things in all dialogues: the subject, the object, and
    “Quality”, which happens between the two.

    Quality is always between the observer and the observed.
  \end{frame}
  \begin{frame}{Splitting and Reassembling}
    All things can be split into smaller compontents, walnuts and protons alike.

    You have to know whether you’re dealing with walnuts or with protons to find
    the scale of your solution.
  \end{frame}
  \begin{frame}{Do abstractions have to hide their details?}
    If everything is made up of components, should we hide the joints and welds
    in our structures from the user?

    The common answer is “yes”, abstractions abstract.
  \end{frame}
  \begin{frame}{The curious case of Git}
    Git is the ultimate leaky abstraction.
  \end{frame}
  \begin{frame}{The curious case of Git}
    It gives you control over the “porcelain”, or user-facing facilities, but
    also over the “plumbing”, and it does so gracefully.

    I can \texttt{git commit}, but I can just as well \texttt{git commit-tree
    <tree> -p <parent-commit>}.
  \end{frame}
  \begin{frame}{Do abstractions have to hide their details?}
    The real answer is “not if you do it well”, which translates to “probably
    yes”.
  \end{frame}
  \section{Summary!}
  \begin{frame}{Why do I have to think about these things?}
    Thinking about how to write abstractions will make you reflect about your
    personal æsthetics—like whether you want to use the pretentious “æ” or
    not—and whether what you build actually holds up to your standards.
  \end{frame}
  \begin{frame}{Why do I have to think about these things?}
    Reading philosophy will inspire you, or at least build a vocabulary that
    heps you articulate your æsthetics, pretentious or not.
  \end{frame}
  \begin{frame}{Techniques for writing better abstractions, summarized}
    A personal guidebook.
    \begin{itemize}
      \item Don’t fear taking a hard stance if it leads to better order and the
            right abstractions. But be gentle to other people if they disagree.
      \item Have a clear classification system. There should be no overlap
            between different function sets in your API. Remember: it’s about
            clear boundaries.
      \item If the order is intuitive, the abstraction will be intuitive as
            well. But please write documentation anyway.
      \item If you’re unsure about anything, talk to someone else. Bee tee
            dubbs: I’m always happy to chat about these things.
    \end{itemize}
  \end{frame}
  \begin{frame}{A Commentated Reading List I}
    \begin{itemize}
      \item Michel Foucault—The Order of Things: not about abstractions per se,
            but nonetheless exciting!
      \item Robert M. Pirsig—Zen and the Art of Motorcycle Maintenance: one of
            the best books I’ve ever read, stock-full of marvelous writing,
            storytelling, and compelling philosophical ideas.
      \item Christopher Alexander—Notes on the Synthesis of Form: provides a
            beautiful system of judging tradeoffs and implementations of ideas.
      \item Zachary Tellman—Elements of Clojure: talks about names and idioms
            rather than how to build anything in particular, and is thus more
            useful than any other programming book I’ve read; not just for
            Clojure programmers.
      \item Guy Steele—Growing a Language: both a paper and a talk. It explores
            building technical abstractions from first principles, and is
            removed enough from technology to be generally useful.
    \end{itemize}
  \end{frame}
  \begin{frame}
    \begin{itemize}
      \item Douglas R. Hofstaedter—Gödel, Escher, Bach: An Eternal Golden Braid:
            a weird and beautiful voyage into meaning, self-reference, and
            recursion.
      \item The Git Book, Chapter 10.1—Git Internals - Plumbing and Porcelain:
            is the start of a journey into weird Git internals that somehow ends
            up being extremely empowering and beautiful.
    \end{itemize}
  \end{frame}
  \begin{frame}{The End}
    \Huge Thank you!
    \linebreak
    \linebreak
    \tiny Slides at \texttt{https://github.com/hellerve/talks}
  \end{frame}
\end{document}
