\documentclass[aspectratio=169]{beamer}
\usetheme{metropolis}
\title{Abstractions! How do I even?}
\date{\today}
\author{Veit Heller}
\institute{EnthusiastiCon 2019}
\begin{document}
  \maketitle
  \begin{frame}{Agenda}
    \begin{itemize}
      \item Motivation!
      \item Michel Foucault—The Order of Things
      \item Robert M. Pirsig—Zen and the Art of Motorcycle Maintenance
      \item Summary!
    \end{itemize}
  \end{frame}
  \section{Motivation!}
  \begin{frame}{Technophilosophy?}
    We’re going to talk a lot about philosophy in this talk. None of it will be
    about the branches of philosophy that seem most notably intertwined with
    our profession: technophilosophy.
  \end{frame}
  \begin{frame}{Abstractions}
    We build abstractions, always.
  \end{frame}
  \begin{frame}{Abstractions}
    So do philosophers; they have a headstart of just a few millenia.
  \end{frame}
  \begin{frame}{Technophilosophy!}
    In a sense, all philosophy is philosophy about technology, if the goal of
    technology is abstracting things.
  \end{frame}
  \begin{frame}{But I don’t have a PhD!}
    Me neither, and English isn’t my first language either.

    The beauty about a lot of philosophy is that it is very simple, because it
    deals with very simple things—but not easy things!
  \end{frame}
  \begin{frame}{Why do I have to think about these things?}
    Because you work on them all day! And because it’s fun!
  \end{frame}
  \section{Michel Foucault—The Order of Things}
  \begin{frame}{Jorge Luis Borges}
    “[\ldots] animals are divided into: (a) belonging to the Emperor, (b)
    embalmed, (c) tame, (d) sucking pigs, (e) sirens, (f) fabulous, (g) stray
    dogs, (h) included in the present classification, (i) frenzied, (j)
    innumerable, (k) drawn with a very fine camelhair brush, (l) \textit{et
    cetera}, (m) having just broken the water pitcher, (n) that from a long
    way off look like flies [\ldots]”

    — Michel Foucault, citing Jorge Luis Borges, in “The Order of Things”
  \end{frame}
  \begin{frame}{Order}
    It is often better to have no system to have the wrong system.
  \end{frame}
  \begin{frame}{Order}
    Classifications are defined by the space between them, not by their label.
  \end{frame}
  \begin{frame}{Order}
    Order is statist: you have to impose rules, and all of them are arbitrary.
  \end{frame}
  \begin{frame}{What makes order good? What makes it bad?}
    Good abstractions impose a sensible order. The space between concepts is
    well defined.
  \end{frame}
  \begin{frame}{Why order?}
    If the order is intuitive, the abstraction will be intuitive as well.
  \end{frame}
  \begin{frame}{Why abstractions?}
    A good abstraction is good because it imposes order. Good order is good
    because it provides abstraction.

    This is a tautology, but it makes sense because this is how the mind
    operates: always classifying, putting in boxes, hiding details.
  \end{frame}
  \section{Robert M. Pirsig—Zen and the Art of Motorcycle Maintenance}
  \begin{frame}{Splitting and reassembling things}
  \end{frame}
  \begin{frame}{Lateral and literal thinking}
  \end{frame}
  \begin{frame}{Do abstractions have to hide their details?}
  \end{frame}
  \begin{frame}{The curious case of Git}
  \end{frame}
  \section{Summary!}
  \begin{frame}{Why should I care, revisited}
  \end{frame}
  \begin{frame}{Techniques for writing better abstractions, summarized}
  \end{frame}
  \begin{frame}{A Commentated Reading List}
    \begin{itemize}
      \item Michel Foucault—The Order of Things: not about abstractions per se,
            but nonetheless exciting!
      \item Robert M. Pirsig—Zen and the Art of Motorcycle Maintenance: one of
            the best books I’ve ever read, stock-full of marvelous writing,
            storytelling, and compelling philosophical ideas.
      \item Christopher Alexander—Notes on the Synthesis of Form: provides a
            beautiful system of judging tradeoffs and implementations of ideas.
      \item Zachary Tellman—Elements of Clojure: talks about names and idioms
            rather than how to build anything in particular, and is thus more
            useful than any other programming book I’ve read; not just for
            Clojure programmers.
      \item Guy Steele—Growing a Language: both a paper and a talk. It explores
            building technical abstractions from first principles, and is
            removed enough from technology to be generally useful.
    \end{itemize}
  \end{frame}
  \begin{frame}{The End}
    \Huge Thank you!
    \linebreak
    \linebreak
    \linebreak
    \small Questions?
    \linebreak
    \linebreak
    \tiny Slides at \texttt{https://github.com/hellerve/talks}
  \end{frame}
\end{document}
